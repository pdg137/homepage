\input amstex

\font\big = cmr14
\centerline{\big Werthwein family tree}
\centerline{January 16, 2003}

\parskip = 5pt

This is a translation of the Werthwein family tree chart, written in German
in the old cursive script, in the possession of Phyllis Ream, by Dan and Paul
Grayson.  I have gone to the local Mormon genealogical library and ordered
the microfilms for Knittlingen to confirm the names and dates provided by the
family tree chart; those films are made fromt he backup copies in the
archives in Stuttgart.  Another thing we could do is to visit Knittlingen to
look at the original church books.

The inscription on the top of the family tree reads ``The family tree of
the Weiss, Haug, D\"urw\"achter, and Werthwein'' families.''

The inscription on the left reads ``C??? Kimball Consul of U. States of
America, June 16, 1886'', but was apparently written by a German, since
``America'' and ``June'' are written in German.

The inscriptions on the bottom read ``The conformity of this information with the information of
the church books of the town of Knittlingen, June 11, 1886, is royally authenticated by the
W\"urtemberg Deacon's office in the town of Dieterle'' and ``The validity of the signature of the
judicial office is certified by the judicial upper office of the upper office in the town of
Maulbron.''  The towns of Maulbronn and Knittlingen are 30 miles west-northwest of Stuttgart,
Germany, and are just about 3 miles from each other.

Now we turn to the individual entries.

% 1
{\bf Bernhard Friederich Weiss}: born December 7, 1742; citizen and barrel
maker here; died December 4, 1777, here.  He married {\bf Maria Elizabetha
Plag} February 13, 1765.

% 2
{\bf Maria Elizabetha Plag}: born June 8, 1748; she married second {\bf
  Ferdinand Christian Haug} July 13, 1779.

% 3
{\bf Ferdinand Christian Haug}: born November 16, 1752; master barrel maker here; [it's hard to
translate this next bit] is the statement to Jakob Friedrich D\"urw\"achter{} listed below with the
``Kappanisten''; emigrated to America in 1805 with his wife.

The children of {\bf Bernhard Friederich Weiss} and {\bf Maria Elizabetha
  Plag} are:
{\bf Maria Magdalina Weiss}, born August 8, 1766; died July 3, 1772;
{\bf Georg Friederich Weiss}, born here May 12, 1768; died May 4, 1778;
{\bf Ana Maria Weiss}, details below; and
{\bf Jakobina Chathrina Weiss}, born April 12, 1775; it's unknown what became of her; (the middle
  name is probably a misspelling of Katharina).

{\bf Ana Maria Weiss} was born April 11, 1771.  
%
Her first marriage was to {\bf Johan Jakob D\"urw\"achter}, February 13, 1794; he was
born February 8, 1746, he was a shoe maker, and he died here January 24, 1806.  
%
Her second marriage was to {\bf Johannes Bauer} October 28, 1806; SOMETHING in
the town of Otisheim, died December 8, 1824; the children of Johannes Bauer
and Ana Maria Weiss died soon after birth.  
%
She died August 20, 1852, in Ruith.

The children of {\bf Ana Maria Weiss} and {\bf Johan Jakob D\"urw\"achter} are:
\item -
{\bf Jakobina Friederika D\"urw\"achter}, born January 16, 1797.  It is unknown what
became of her; there is no annotation about it.  She wasn't one of the
confirmands in 1811, which might signify emigration through participation in the
separatist movement;
\item -
{\bf Rosina Friderika D\"urw\"achter}, details below;
\item -
{\bf Jakob Friederich D\"urw\"achter}, born January 14, 1804; (twin); emigrated and lives in
America, in Philadelphia; and
\item -
{\bf SOMETHING Jakobina D\"urw\"achter}, born January 14, 1804, (twin); it's unknown what
became of her, as she wasn't one of the confirmands in 1818.

% 13 
{\bf Rosina Friederika D\"urw\"achter} was born July 10, 1799.  
She married {\bf Philipp Jakob Werthwein} June 5, 1823; 
He was born March 9, 1792; was a smith, and died here February 5, 1840.
They had seven children:
{\bf Jakob Eberhard Werthwein}, born April 10, 1824; lives in Hampshire, Illinois, North America;
{\bf Louisa Wilhelmina Werthwein}, born April 22, 1825; lives in Chicago;
{\bf Wilhelmina Katharine Werthwein}, born August 15, 1826; lives in
Newark, New Jersey, North America;
an {\bf unnamed child} born December 29, 1827, who died one day later;
an {\bf unnamed child} born December 30, 1829, who died the same day;
{\bf Christina Rosina Werthwein}, born December 1830, lives in Chicago; and 
an {\bf unnamed child} born April 12, 1832, who died the same day.
Her second marriage was to {\bf Johannes Wolf} December 22, 1840.  
He was a miller in Ruith; died December 26, 1874.
They had one child:
{\bf Katharina Elisabeth Wolf}, born May 4, 1842, lives in Chicago.

So much for the family tree chart.

Here is some further information from Phyllis Ream.

Phyllis thinks {\bf Christina Rosina Werthwein} married someone important in Chicago.

{\bf Jakob Eberhard Werthwein} (known simply as Eberhard) emigrated to America before marriage.  He
and his family are buried in the cemetery in Hampshire, Illinois; we should photograph the stones.
His son {\bf Charles Werthwein} married {\bf Katherine Becker}, born 1864 in Jacksonville,
Illinois.  They met when he saw an attractive photograph in the window of a photography studio, and
she emerged from the studio.  She lived next door to the Mary Todd family who married Abraham
Lincoln.

Charles and Katherine had four children.  Two died early.  The other two were {\bf Pearl Werthwein}
(never married, had wonderful operatic voice, family sent her to Europe for training at the Sorbonne, but
she had some problem with her vocal cords and an operation ended her singing career; 
% egotistical woman, doted on herself, 
she died in Hampshire) and {\bf Percy C.{}
Werthwein}, born May 14, 1882, in Hampshire, Illinois.  Attended Univ. of Wisconsin until he withdrew so that
the family could afford to send his sister to Europe.  The ``C'' was simply an initial, 
with no associated name.

Percy married {\bf Alice Caroline Bruner} November 4, 1911, in Chester, Iowa.  She was born June
11, 1890, in Elberon, Iowa, to Adam Bruner and Anna Fromm, and she died February 18, 1984, in
National City, California.  They went out to South Dakota and homesteaded there.  Then they
returned to Hampshire.

Katherine died in Hampshire, June 20, 1948, in Hampshire.

Percy died November 18, 1935, in Waterman, Illinois, as a result of an automobile accident near
Leland.  He was on a hunting expedition with friends.

There is a booklet called ``Kane County, 1818-1968, Illinois Sesquicentennial'', with handwritten
marginal notes by Alice, which has a drawing of the Werthwein farm (in Burlington township) on page
34, center left, labelled as the residence of Henry Weightman.  The cupola pictured there on the
barn had to be removed when they moved there in 1915.  They bought the farm from the brothers Henry
and William Ackerman and \$141 per acre.  Phyllis and her siblings were all born in the farmhouse
there.  One of Alice's notes says that ``W. Grandpa'' (Charles), Bill, Alice, and Dolly attended
the Elgin Academy.

The children of Percy and Alice are:
{\bf Carl Burton Werthwein}, born 1912 in Chester, Iowa, died May 3, 1984;
{\bf Dwight Werthwein}, born in Burlington township, Illinois, deceased,
{\bf Roger Percy Werthwein}, born in Burlington township, Illinois, deceased;
{\bf Greta Anne Werthwein}, born in Burlington township, Illinois, married name Parks, living in
Stella, North Carolina;
{\bf Mark Werthwein}, born in Burlington township, Illinois, living in Sycamore, Illinois;
{\bf George Werthwein}, born in Burlington township, Illinois, lives in Oceanside, California;
{\bf Alice Caroline Werthwein}, born in Burlington township, Illinois, lives in Chula Vista, California; and
{\bf Phyllis Pearl Werthwein}, born in Burlington township, Illinois, who married Robert Ream, her
seventh cousin, somehow (the documentation for that relationship is somewhere in Phyllis' house).

\bye

